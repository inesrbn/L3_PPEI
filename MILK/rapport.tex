\documentclass[a4paper,11pt]{article}

\usepackage[francais]{babel}
\usepackage[no-math]{fontspec}

%%
%%

\title{La Prévention des Cyberattaques}
\author{MILK : Moeez, Ines, Louison, Hugo}
\date{2023/2024}

\begin{document}

\maketitle

\begin{abstract}
  La cybersécurité est un enjeu majeur pour les entreprises et les particuliers. 
  Les cyber-attaques sont de plus en plus fréquentes et sophistiquées, ce qui rend la prévention des attaques de plus en plus difficile. 
  Cependant, il existe des mesures que vous pouvez prendre pour réduire les risques d’attaques.
\end{abstract}

%%
%%

\section{Introduction}

Dans cette partie, nous allons définir ce qu’est la cybersécurité et pourquoi elle est importante. 
Nous allons également discuter des différents types de cyber-attaques et des risques qu’elles représentent.

%%
%%

\section{Partie 1: Mesures de prévention}

Dans cette partie, nous allons discuter des mesures que l’on peut prendre pour prévenir les cyber-attaques, on a par exemple : 

  - Utilisation de mots de passe forts

  - Sauvegarde régulière des données

  - Mises à jour de sécurité sur tous les appareils

  - Utilisation d’un antivirus

  - Téléchargement d’applications uniquement sur les sites officiels

  - Méfiance envers les messages inattendus

  - Vérification des sites sur lesquels vous faites des achats

  - Maîtrise des réseaux sociaux

  - Séparation des usages personnels et professionnels

  - Évitement des réseaux WiFi publics ou inconnus

%%
%%

\section{Partie 2: Les bonnes pratiques en entreprise}

Dans cette partie, nous allons discuter des bonnes pratiques en entreprise pour prévenir les cyber-attaques. Nous allons aborder les sujets suivants :

  - Investissement dans de bons outils

  - Installation d’antivirus efficaces

  - Sensibilisation, formation et information du personnel

%%
%%

\section{Partie 3: Que faire en cas d’attaque ?}

Dans cette partie, nous allons discuter des mesures à prendre en cas d’attaque. Nous allons aborder les sujets suivants :

  - Conseils et assistance pour les victimes de cyber malveillance

  - Notification de violation de données personnelles

  - Contact avec la police

%%
%%

\section{Conclusion}

%%
%%

\bibliographystyle{plain}
\bibliography{rapport}

\end{document}
