\documentclass[a4paper,11pt]{article}

\usepackage[francais]{babel}
\usepackage[no-math]{fontspec}

%%
%%

\title{La Prévention des Cyberattaques}
\author{MILK : Moeez, Ines, Louison, Hugo}
\date{2023/2024}

\begin{document}

\maketitle

\begin{abstract}
  La cybersécurité est un enjeu majeur pour les entreprises et les particuliers. 
  Les cyber-attaques sont de plus en plus fréquentes et sophistiquées, ce qui rend la prévention des attaques de plus en plus difficile. 
  Cependant, il existe des mesures que vous pouvez prendre pour réduire les risques d’attaques.
\end{abstract}

%%
%%

\section{Introduction}

Dans cette partie, nous allons définir ce qu’est la cybersécurité et pourquoi elle est importante. 
Nous allons également discuter des différents types de cyber-attaques et des risques qu’elles représentent.

%%
%%

\section{Partie 1: Mesures de prévention}

Dans cette partie, nous allons discuter des mesures que l’on peut prendre pour prévenir les cyber-attaques, on a par exemple : 

  - La majorité des attaques est souvent due à des mots de passe trop simples ou réutilisés. Il faut souvent les changer et avoir un mot de passe longs et différents pour chaque compte, sites où l’on doit se connecter. Ainsi les pirates informatiques ont plus de mal à décoder le mot de passe et ont beaucoup moins de chance de le découvrir. Plus le mot de passe est long et complexe, plus il met de temps à être déchiffré. On peut utiliser un gestionnaire de mot de passe qui permet de centraliser les mots de passe dans une base de données et aussi activer l’option de la double authentification qui permet de renforcer la sécurité.

  - La sauvegarde peut être un bon moyen d’éviter les cyber-attaques, car en cas de panne, de vol ou de perte d’un appareil on peut retrouver les informations perdues.

  - Avoir les appareils à jour est un bon point, de ce fait, on corrige les éventuelles failles de sécurité s’il y en a une ou plusieurs ce qui éviterait aux personnes malveillantes de s’introduire dans nos appareils, ce qui pourrait endommager ou causer la perte de nos données sensibles.

  - Le plus connu de tous, l’antivirus est un bon moyen de protection, il existe gratuitement ou de manière payante selon les usages de l’utilisateur et le niveau de protection de données requise. Aussi, pour vérifier que l’on a pas été infecté d’un virus on peut régulièrement faire des mises à jour et des analyses avec l’antivirus


  - En terme de téléchargement d’applications, vidéo, streaming etc.. Le mieux est de le faire sur des sites ou plateformes, officielle uniquement, cela limite grandement les risques d’une installation piégée qui pourrait pirater ou installer des virus frauduleux sur nos appareils.

  - Méfiance envers les messages inattendus

  - Vérification des sites sur lesquels vous faites des achats

  - Maîtrise des réseaux sociaux

  - Séparation des usages personnels et professionnels

  - Évitement des réseaux WiFi publics ou inconnus

%%
%%

\section{Partie 2: Les bonnes pratiques en entreprise}

Dans cette partie, nous allons discuter des bonnes pratiques en entreprise pour prévenir les cyber-attaques. Nous allons aborder les sujets suivants :

  - Investissement dans de bons outils

  - Installation d’antivirus efficaces

  - Sensibilisation, formation et information du personnel

%%
%%

\section{Partie 3: Que faire en cas d’attaque ?}

Dans cette partie, nous allons discuter des mesures à prendre en cas d’attaque. Nous allons aborder les sujets suivants :

  - Conseils et assistance pour les victimes de cyber malveillance

  - Notification de violation de données personnelles

  - Contact avec la police

%%
%%

\section{Conclusion}

%%
%%

\bibliographystyle{plain}
\bibliography{rapport}

\end{document}
