\documentclass[a4paper,11pt]{article}

\usepackage[T1]{fontenc}
\usepackage[french]{babel}

%%
%%

\begin{document}
\title{\textbf{\huge{Prévention des Cyberattaques}}}
\author{MILK : Moeez, Ines, Louison, Hugo}
\date{2023/2024}
\maketitle

\def\contentsname{Sommaire}
\tableofcontents

\begin{abstract}
  Nous allons dans plusieurs partie de ce rapport traiter au sujet des cyberattaques, leurs fonctionnement, les risques associés selon le type d'attaque et les dif{\null}férentes manières de nous en prémunir.
\end{abstract}

%%
%%

\section{Introduction}

La cybersécurité est un enjeu majeur pour les entreprises et les particuliers. 
Les cyber-attaques sont de plus en plus fréquentes et sophistiquées, ce qui rend la prévention des attaques de plus en plus dif{\null}f{\null}icile. 
Cependant, il existe de nombreuses mesures que l'on peut prendre pour réduire les risques d'attaques et de réussite de ces dernière.
\\Dans ce rapport, nous allons déf{\null}inir ce qu'est la cybersécurité et pourquoi elle est importante. 
Nous allons également discuter des dif{\null}férents types de cyber-attaques et des risques qu'elles représentent et quelles mesures préventives nous pouvons prendre pour se préparer contre celles-ci.

%%
%%

\section{Partie 1: Mesures de prévention}

La cybersécurité est un ensemble de processus, d'outils et de cadres visant à protéger les réseaux, les appareils, les programmes et les données des cyberattaques~\cite{article12}.
Les cybercriminels lancent de telles attaques pour obtenir un accès non autorisé à des systèmes informatiques, interrompre des opérations d'entreprise, modif{\null}ier, manipuler ou voler des données, réaliser de l'espionnage industriel ou extorquer de l'argent aux victimes. Les cyberattaques af{\null}fectent actuellement un grand nombre d'individus et d'entreprises chaque année, sachant qu'une attaque a lieu en moyenne toutes les 39 secondes~\cite{article1}.
\\Les cyberattaques peuvent causer des préjudices f{\null}inanciers ou une atteinte à la réputation, endommager l'infrastructure informatique et entraîner des amendes réglementaires.
Pour protéger leurs précieuses ressources et données des pirates, les entreprises et les individus ont besoin d'un solide dispositif de cybersécurité. En 2021, le cybercrime a coûté au monde environ 6 000 milliards d'euros. D'ici 2025, ce coût passera à 10 500 milliards d'euros~\cite{article1}.
Le cybercrime est un problème de plus en plus sérieux, et pour s'y attaquer, il est essentiel de disposer d'un excellent dispositif de cybersécurité. Les individus, les gouvernements, les entreprises, les organismes à but non lucratif et les établissements d'enseignement risquent tous de subir des cyberattaques et des violations de données.
À l'avenir, le nombre d'attaques se multipliera avec l'évolution des technologies numériques, l'augmentation du nombre d'appareils et d'utilisateurs ainsi que le rôle de plus en plus stratégique des données dans le monde de l'économie et du numérique.
Pour minimiser les risques d'attaque et pour sécuriser les systèmes et les données, il est plus que recommandé d'avoir un solide dispositif de cybersécurité et de le garder le plus à jour possible.

%%
%%

\subsection{Les différents types de cyberattaque et quels sont les risques qu'elles représentent.}

Une cyberattaque est une action malveillante visant à compromettre la sécurité d'un système informatique, d'un réseau ou de données~\cite{article3,Article10}. Les cyberattaques peuvent être motivées par diverses raisons : f{\null}inancières, idéologiques, politiques ou même personnelles.
Il existe plusieurs types de cyberattaques, chacunes ayant des objectifs et des méthodes d'action dif{\null}férentes.
Les types de cyberattaques les plus courantes sont généralement : l'attaque par déni de service ou DDoS, l'attaque Man-in-the-middle ou MITM, l'attaque par phishing, l'attaque par ransomware, l'attaque par injection SQL , l'attaque par force brute ou encore l'attaque par exploitation de vulnérabilités~\cite{article2}.
\\L'attaque DDoS vise à submerger les ressources d'un système jusqu'à ce qu'il soit incapable de répondre à des demandes de service légitimes. Une attaque DDoS survient quand un grand nombre de machines hôtes sont infectées par des malwares et contrôlées par l'assaillant. Les attaques par déni de service sont dif{\null}férentes des autres types de cyberattaques qui permettent au hacker d'obtenir l'accès à un système ou d'augmenter l'accès dont il dispose actuellement. Avec ces types d'attaques, l'assaillant retire un bénéf{\null}ice immédiat de ses ef{\null}forts. L'objectif des DDoS est simplement d'interrompre l'ef{\null}f{\null}icacité du service de la cible.
\\L'attaque MITM permet à un assaillant d'écouter les données envoyées entre deux personnes, réseaux ou ordinateurs. L'attaquant intercepte les communications entre les deux parties sans que ni l'une ni l'autre ne s'en aperçoive et peut même envoyer des données en se faisant passer pour l'expéditeur légitime de celles-ci.
\\L'attaque par phishing consiste à envoyer un e-mail frauduleux qui semble provenir d'un proche ou une connaissance pour inciter la victime à fournir des informations personnelles telles que des mots de passe ou des informations bancaires.
\\L'attaque par ransomware consiste à chif{\null}frer des données importantes pour la victime sur un ordinateur ou un réseau et à exiger une rançon pour les déchif{\null}frer en menaçant l'exposition de celles- ci, ce qui incite grandement à payer la rançon et espérer récupérer les données~\cite{article9}.
\\L'attaque par injection SQL se produit quand un utilisateur malveillant envoie une entrée qui modif{\null}ie la requête SQL envoyée à la base de données, ce qui permet d'exécuter des requêtes non prévues à l'origine par l'application web. De cette façon l'assaillant peut récupérer des données sensibles qui ne doivent normalement pas être communiquées aux utilisateurs~\cite{article7}. 
\\L'attaque par force brute fonctionne sur un principe qui est d'essayer toutes les combinaisons possibles de mots de passe jusqu'à trouver celui qui correspond.
\\L'attaque par exploitation de vulnérabilités consiste à exploiter une vulnéra-bilité connue dans un système pour y accéder ou y exécuter du code malveillant.
\\Les risques associés aux cyberattaques sont nombreux et peuvent varier en fonction du type d'attaque et du système ciblé. Les risques les plus courants sont la perte ou le vol de données sensibles, la perturbation des opérations commerciales, la perte f{\null}inancière, la violation de la vie privée et la perte de réputation.

%%
%%

\section{Partie 3: Que faire en cas d’attaque ?}

Une cyberattaque doit être gérée avec méthode et de l’organisation afin d’en limiter les impacts et de permettre une reprise d’activité dans les meilleurs délais et conditions de sécurité pour éviter une récidive.
La première chose à faire est de débrancher la machine d’Internet ou du réseau informatique.Cette mesure permet de stopper la propagation de l’attaque et de protéger les autres appareils connectés au même réseau.
Il est essentiel de ne pas éteindre l’appareil touché par l’attaque. En effet, certains éléments de preuve contenus dans la mémoire de l’équipement pourraient être effacés si l’appareil était éteint. Les experts en sécurité informatique auront besoin de ces éléments pour analyser l’attaque et déterminer la meilleure stratégie de réponse.
Ensuite il faut alerter immédiatement le support informatique. Ce dernier sera en mesure de prendre les mesures nécessaires pour contenir ou réduire les conséquences de la cyberattaque.

Il est important de ne plus toucher à l’appareil qui pourrait être compromis afin d’éviter de supprimer des traces utiles pour les investigations à venir. 
Il faut informer collègues de la situation pour qu’ils prennent les mesures nécessaires pour protéger leur propre équipement. En effet, une mauvaise manipulation de la part d’un autre collaborateur pourrait aggraver le contexte.
Il ne faut surtout pas donner de rançon. Cela encourage les cybercriminels à chercher à attaquer à nouveau pour financer leur activité criminelle tout en n’ayant aucune garantie qu’ils tiendront leur parole.
Il faut alerter la banque au cas où des informations permettant de réaliser des transferts de fonds auraient pu être dérobées.
Déposez plainte avant toute action de remédiation en fournissant toutes les preuves en votre possession.
Identifiez l’origine de l’attaque et son étendue afin de pouvoir corriger ce qui doit l’être et éviter un nouvel incident.
Notifiez l’incident à la CNIL dans les 72 h si des données personnelles ont pu être consultées, modifiées ou détruites par les cybercriminels.
%%
%%

\section{Conclusion}

La cybersécurité est un sujet important qui doit être pris au sérieux. En prenant les mesures appropriées, vous pouvez réduire considérablement le risque d’attaques. Cependant, il est également important d’être préparé en cas d’attaque. En suivant les bonnes pratiques et en étant vigilant, vous pouvez protéger vos données et votre entreprise contre les cyber-menaces.

%%
%%

\bibliographystyle{plain}
\bibliography{rapport}

\end{document}
