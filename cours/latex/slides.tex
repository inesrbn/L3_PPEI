\documentclass[aspectratio=169]{beamer}

\usepackage[francais]{babel}
\usepackage[no-math]{fontspec}
\usepackage{textpos}
\usepackage{amssymb}
\usepackage{relsize}
\usepackage{xspace}
\usepackage{etoolbox}
\usepackage{mdi}
\usepackage{multicol}

%%
%% beamer
%%

\usetheme{stripes}
\setbeamercovered{invisible}
\AtBeginSection[]{
  \begin{frame}<beamer>{Sommaire}
    \setcounter{tocdepth}{2}
    \tableofcontents[currentsection]
  \end{frame}}

\renewcommand\insertheadline{%
  \insertheadlinepart{.5}{left}{white}{\insertshortauthor
    \edef\SECTION{{\insertsection}}%
    \expandafter\ifblank\SECTION{}{ / \insertsection}}
  \insertheadlinepart{.5}{right}{white}{{\insertframenumber} / {\inserttotalframenumber}}
}

%%
%% LaTeX examples
%%

\usepackage{fancyvrb}
\usepackage{pygments}
\usepackage[most]{tcolorbox}

\tcbset{
  size=small,
  colback=yellow!50!gray!10,
  colframe=yellow!50!black!20,
}

\directlua{os.execute('mkdir -p src')}

\usepackage{sverb}
\newwrite\codewrite

\def\gedef{\global\edef}

\newcommand\pdf[2][]{{\def\IG{\tcbincludegraphics[#1]}%
  \expandafter\IG\csname pdf:#2\endcsname}}

\newcommand\pyg[1]{\expandafter\input\csname pyg:#1\endcsname}

\newcommand\pygpdf[2][\columnwidth]{%
  \begin{multicols}{2}
    \pyg{#2}
    \columnbreak
    \pdf[width=#1]{#2}
  \end{multicols}}

\newenvironment{latex}[2][tpl=standalone]{
  \immediate\openout\codewrite src/tex-#2.opt
  \immediate\write\codewrite{#1}%
  \immediate\closeout\codewrite
  \edef\codename{src/tex-#2}%
  \expandafter\verbwrite{src/tex-#2.src}
}{\endverbwrite
  \directlua{os.execute('python mk.py \codename')}%
  \expandafter\input\codename.out
}

\newenvironment{template}[1]{
  \edef\codewritename{{src/tpl-#1.tex}}%
  \expandafter\verbwrite\codewritename
}{\endverbwrite}

%%
%% LaTeX templates
%%

\begin{template}{standalone}
  \documentclass[varwidth{{=TEXTWIDTH}}]{standalone}
  \usepackage[francais]{babel}
  \usepackage{graphicx}
  \usepackage{tikz}
  \pagestyle{empty}
  \parindent=0pt
  \begin{document}
  {{TEX}}
  \end{document}
\end{template}

\begin{template}{article-crop}
  \documentclass{article}
  \usepackage[francais]{babel}
  \usepackage{graphicx}
  \usepackage{tikz}
  \pagestyle{empty}
  \parindent=0pt
  {{\textwidth=TEXTWIDTH}}
  \begin{document}
  {{TEX}}
  \end{document}
\end{template}

%%
%% macros
%%

\colorlet{minorcolor}{stripesdarkblue!50}

\newcommand\minor[1]{{\color{minorcolor}#1}}
\let\URL=\url
\renewcommand\url[1]{{\color{stripesdarkblue}\URL{#1}}}
\newcommand\pix[2][]{{\tcbset{opacityback=0,opacityframe=0}\tcbincludegraphics[#1]{pix/#2}}}
\renewcommand\>{\ensuremath{\Rightarrow}}
\renewcommand\tt[1]{\texttt{#1}}
\newcommand\BibTeX{\textsc{Bib}\!\TeX}
\newcommand\macro[1]{\tt{\PY{k}{\PYZbs{}#1}}}
\newcommand\TikZ{Ti\emph{k}Z}

%%
%% slides
%%

\begin{document}

\title{\LaTeX}
\subtitle{une courte introduction}
\author{Franck Pommereau}
\institute{Université Évry Val d'Essonne / Paris-Saclay}
\date{L3 informatique}

\begin{frame}
  \titlepage
  \centerline{\pix[height=2cm]{logo-ueve-upsay}}
\end{frame}

%%
%%
%%

\section{Quoi? Pour qui? Pour quoi?}

%%

\begin{frame}{Qu'est-ce que \LaTeX?}
  \begin{itemize}
    \item un \alert{traitement de texte} au sens propre
      \begin{itemize}
        \item \emph{what you \underline{think} is what you get}
          \> balisage logique des documents
        \item mise en page \alert{automatique}
          \> on se concentre sur le fond
          \hfill\minor{(on laisse la forme au logiciel)}
        \item \alert{qualité typographique exceptionnelle}
      \end{itemize}
    \item un langage de programmation
      \begin{itemize}
        \item invisible le plus souvent
        \item virtuellement illimité
        \item nombreuses \alert{tâches simplifiées}
      \end{itemize}
    \item une communauté
      \begin{itemize}
        \item des milliers d'extensions librement disponibles
        \item abondance de documentation
        \item aide disponible: collègues, forums, FAQ, \dots
      \end{itemize}
  \end{itemize}
\end{frame}

%%

\begin{frame}{Historique}
  \begin{description}
    \item[1977] Donald Knuth développe \alert{{\TeX}} ($\tau\epsilon\chi$)
      \begin{itemize}
        \item bases du langage de programmation
        \item moteur typographique
        \item système de fontes
        \end{itemize}
    \item[1985] Leslie Lamport développe {\LaTeX}
      \begin{itemize}
        \item un ensemble d'\alert{extensions à \TeX}
        \item \alert{simplifie} de nombreuses tâches
        \item renforce la structure des documents
        \item \alert{séparation entre structure logique et rendu graphique}
      \end{itemize}
    \item[1990\dots] {pdf\LaTeX} \> sortie en PDF
    \item[2007] {lua\LaTeX} \hfill\minor{(version recommandée)}
      \begin{itemize}
        \item support d'Unicode \hfill\minor{(cf. juste après)}
        \item support des fontes modernes
        \item programmation en \tt{lua}
      \end{itemize}
    \end{description}
\end{frame}

%%

\begin{frame}[t]{Interlude}{Codages du texte}
  \begin{textblock}{5}(9,-2)
    \pix[size=tight,opacityframe=1,scale=1.2,colframe=gray]{martine-utf-8}
  \end{textblock}
  \begin{itemize}
  \item ASCII
    \minor{états-unien}
  \item ASCII étendus
    \minor{langues européennes}
  \item Unicode
    \minor{tout (et n'importe quoi)}
    \begin{itemize}
    \item multiples encodages
    \item UTF8 se standardise
      \minor{compatible ASCII}
    \end{itemize}
  \end{itemize}
\end{frame}


%%

\begin{frame}{Ressources}
  \begin{itemize}
  \item \url{http://www.latex-project.org}\\
    site de référence sur \LaTeX, avec les instructions d'installation
  \item \url{http://www.ctan.org}\\
    \emph{the comprehensive {\TeX} archive network}
  \item \url{http://www.gutenberg.eu.org}\\
    groupe francophone des utilisateurs de \TeX, {\LaTeX}
    \hfill\minor{(doc en français)}
  \item \url{http://www.xm1math.net/texmaker}\\
    éditeur multi-plateformes pour \LaTeX, avec toutes les aides nécessaires
  \item \url{http://www.overleaf.com}\\
    éditeur en ligne pour \LaTeX, avec partage de documents
    \hfill\minor{(plus pas mal de doc)}
  \item \url{http://tex.stackexchange.com}\\
    forum d'aide sur \LaTeX
  \item \url{http://www.texample.net}\\
    des exemples de toutes sortes
  \end{itemize}
\end{frame}

%%

\begin{frame}{Quand préférer \LaTeX?}
  \begin{textblock}{5}(12.5,-1.5)
    \pix[width=25mm]{sd-card}
  \end{textblock}
  \begin{itemize}
    \item documents \alert{complexes}
      \begin{itemize}
        \item tables et sommaires, références croisées, \dots
        \item formules mathématiques
      \end{itemize}
    \item documents \alert{longs}
      \begin{itemize}
        \item thèses, mémoires, livres, etc. \hfill\minor{(ma thèse $+$ mon HDR $<$ 1Mo)}
        \item mise en page et traitements rapides
      \end{itemize}
    \item documents \alert{durables et portables}
      \begin{itemize}
        \item compatibilité dans le temps et entre ordinateurs
        \item format lisible sans logiciel particulier
        \item pas besoin d'un ordinateur puissant
      \end{itemize}
    \item documents \alert{soignés}
  \end{itemize}
  \vfill
  \begin{block}{Quand s'en passer?}
    \begin{itemize}
      \item mises en pages très visuelles
        \hfill\minor{(ces diapos sont faites avec \LaTeX)}
      \item documents jetables
    \end{itemize}
  \end{block}
\end{frame}

%%

\begin{frame}{À qui s'adresse \LaTeX?}
  \begin{itemize}
  \item les informaticiennes
  \item les mathématiciens
  \item les scientifiques en général
  \item \alert{quiconque souhaite concevoir et produire ses documents}
  \item quiconque est prêt à l'effort initial
    \begin{itemize}
    \item la plupart des tâches sont simples à faire
    \item ça se complique seulement si on veut du compliqué
    \end{itemize}
  \end{itemize}
  \vfill
  \begin{block}{À qui le déconseiller?}
    \begin{itemize}
    \item les impatientes \minor{(et qui ont du temps à perdre)}
    \item ceux qui n'en ont pas besoin \minor{(ou qu'on ne veut pas
      aider plus tard)}
    \end{itemize}
  \end{block}
\end{frame}

%%
%%
%%

\section{Les bases}

%%

\begin{frame}{Édition et compilation}
  \begin{center}
    \begin{tikzpicture}[xscale=7,yscale=-3]
      \node(tex) at (0,0) {\scalebox{6}{\mdi{file-document-outline}}};
      \node(pdf) at (1,0) {\scalebox{6}{\mdi{file-image-outline}}};
      \node(aux) at (1,1) {\color{minorcolor}\scalebox{6}{\mdi{file-multiple-outline}}};
      \draw[->](tex)--node[above]{\relsize{1}\tt{lualatex mondoc}}(pdf);
      \node[left] at (tex.west) {\tt{mondoc.tex}};
      \node[right] at (pdf.east) {\tt{mondoc.pdf}};
      \node[right,text width=15mm] at (aux.east) {\color{minorcolor}
        \tt{mondoc.aux}\\
        \tt{mondoc.log}\\
        \tt{...}};
    \end{tikzpicture}
    \vfill
    \centerline{(c'est automatisé par les éditeurs)}
  \end{center}
\end{frame}

%%
  
\begin{latex}[tpl=None]{hello-mini}
  \documentclass{article}
  \begin{document}
    Hello world!
  \end{document}
\end{latex}
 
\begin{frame}[fragile]{Hello world!}
  \pygpdf[.33\textwidth]{hello-mini}
\end{frame}

%%

\begin{latex}[tpl=None]{hello-article}
  \documentclass{article}
  \begin{document}

  \title{Mon premier document}
  \author{Franck Pommereau}
  \date{1 septembre 2023}
  \maketitle

  \section{Hello world!}

  Où je dis bonjour.

  \section{Bye bye world\dots}

  Où je dis au revoir.

  \end{document}
\end{latex}

\begin{frame}{Hello world!}{Le retour}
  \vspace*{-3mm}
  \begin{multicols}{2}
    \relsize{-1}
    \pyg{hello-article}
    \columnbreak
    \pdf[width=.33\textwidth]{hello-article}
  \end{multicols}
\end{frame}

%%

\begin{latex}[tpl=None]{hello-beamer}
  \documentclass{beamer}
  \begin{document}

  \title{Mon premier document}
  \author{Franck Pommereau}
  \date{1 septembre 2023}
  \maketitle

  \section{Hello world!}

  Où je dis bonjour.

  \section{Bye bye world\dots}

  Où je dis au revoir.

  \end{document}
\end{latex}

\begin{frame}{Hello world!}{Le retour de la vengeance}
  \vspace*{-3mm}
  \begin{multicols}{2}
    \relsize{-1}
    \pyg{hello-beamer}
    \columnbreak
    \pdf[width=\columnwidth]{hello-beamer}
  \end{multicols}
\end{frame}

\begin{latex}[tpl=standalone,textwidth=5cm]{input}
espaces     multiples
     et sauts
   de
lignes = une espace
% commentaire
% ignoré mais utile!

ligne vide = nouveau paragraphe
\end{latex}

\begin{latex}[pdf=False]{catcodes}
  # $ & ~ _ ^ \ { } %
\end{latex}

\begin{frame}{Comment {\LaTeX} lit ce que vous tapez?}
  \pygpdf{input}
  \vfill
  \begin{block}{Caractères spéciaux}
    \pyg{catcodes}
  \end{block}
\end{frame}

%%

\begin{latex}[pdf=False]{macros}
  \documentclass
  \;  \!  \\
\end{latex}

\begin{latex}[pdf=False]{macros-args}
  \title{Mon titre}
\end{latex}

\begin{latex}[pdf=False]{macros-opts}
  \documentclass[12pt]{article}
\end{latex}

\begin{latex}[pdf=False]{envs}
  \begin{center}
    \begin{small}
      texte
    \end{small}
  \end{center}
\end{latex}

\begin{frame}{Commandes et environnements}
  \begin{multicols}{2}
    \begin{itemize}
      \item commandes (macros): \pyg{macros}
      \begin{itemize}
        \item arguments: \pyg{macros-args}
        \item options: \pyg{macros-opts}
      \end{itemize}
  \columnbreak
    \item environnements: \pyg{envs}
      \begin{itemize}
        \item toujours respecter l'imbrication
        \item un bon éditeur de texte aide (l'indentation aussi)
      \end{itemize}
    \end{itemize}
  \end{multicols}
\end{frame}

%%

\begin{latex}[tpl=standalone,textwidth=5cm,obeylines=True]{spaces}
  \dots hello
  \dots{} hello
  {\dots} hello
\end{latex}

\begin{frame}[fragile]{Attention aux espaces!}{après les commandes}
  \pygpdf[.8\columnwidth]{spaces}
\end{frame}

%%

\begin{latex}[pdf=False]{doc}
  \documentclass[OPTIONS]{CLASSE}
    %%% entête (ou préambule)
    % déclarations, configuration, etc. (aucun contenu)
  \begin{document}
    %%% titre, auteur, etc.
    \title{TITRE DU DOCUMENT}
    \author{AUTEURS}
    \date{DATE}
    \maketitle

    %%% corps du document
    % sections, texte, figures, etc.
  \end{document}
\end{latex}

\begin{frame}{Structure d'un document}
  \pyg{doc}
\end{frame}

%%

\begin{frame}{Classes}
  \begin{description}
    \item[article] document court (article de revue ou de conférence)
    \item[report] documents un peu plus long (rapports de recherche)
    \item[book] document long (parties, chapitres, sauts de pages, etc.)
    \item[letter] courrier (salutations, adresses, etc.)
    \item[beamer] présentation à projeter (comme maintenant)
    \item[$\cdots$] classe fournie par un éditeur
  \end{description}
  \begin{block}{Quelques options courantes}
    \begin{description}
      \item[a4paper] pour avoir des pages A4 ($21 \times 29.7$ cm)
      \item[10pt, \dots] taille par défaut du texte
      \item[oneside] pour l'impression recto simple\dots
      \item[twoside] {\dots} ou recto-verso
    \end{description}
  \end{block}
\end{frame}

%%

\begin{latex}[pdf=False]{packages}
  % documents en français
  \usepackage[french]{babel}

  % symboles et fontes mathématiques supplémentaires
  \usepackage{amssymb}

  % documents multi-colonnes
  \usepackage{multicol}

  % inclusion d'images et texte en couleur
  \usepackage{graphicx}
  \usepackage{xcolors}

  % dessin de figures directement dans LaTeX
  \usepackage{tikz}
\end{latex}

\begin{frame}[fragile]{Paquetages d'extensions}
  \pyg{packages}
\end{frame}

%%

\begin{latex}[tpl=standalone,textwidth=52mm]{text}
  texte mis \emph{en valeur}
  ou \textit{en italique},
  \textbf{gras},
  \textbf{gras et
    \textit{italique}},
  \textsc{petites capitales},
  {\small plus petit},
  {\large plus grand},
  {\tiny vraiment très petit},
  {\Large vraiment très grand},
  {\LARGE de plus} {\huge en plus}
  {\Huge grand.}
  \textit{texte correctement
    \emph{mis en valeur}
    dans de l'italique}
\end{latex}

\begin{frame}{Styles de texte}
  \vspace*{-3mm}
  \pygpdf{text}
\end{frame}

%%

\begin{latex}[tpl=standalone,textwidth=5cm,obeylines=True]{ligs}
  fin / f{i}n
  effet / e{f}fet
  efficace / ef{f}icace
  fleur / f{l}eur
  oeuf / {\oe}uf
  encyclopaedia / encyclop{\ae}dia
  tiret simple -
  demi quadratin --
  quadratin ---
  moins $-$
  \'{E} \c{s} \^{z} \r{a} \l
\end{latex}

\begin{frame}{Ligatures et diacritiques}
  \vspace*{-3mm}
  \pygpdf{ligs}
\end{frame}

%%

\begin{latex}[tpl=standalone,textwidth=5cm]{lists}
  \begin{itemize}
  \item un premier item;
  \item suivi d'un second:
    \begin{itemize}
    \item avec une imbrication,
    \item de sous-items.
    \end{itemize}
  \end{itemize}

  \begin{enumerate}
  \item Même principe.
  \item Et là aussi:
    \begin{enumerate}
    \item on peut imbriquer;
    \item comme on veut.
    \end{enumerate}
  \end{enumerate}
\end{latex}

\begin{frame}{Listes et énumérations}
  \relsize{-1}
  \pygpdf{lists}
\end{frame}

%%
%%
%%

\section{Sommaires, références, bibliographie}

%%

\begin{latex}[pdf=False]{sections}
  % seulement dans la classe book
  \part{Titre de partie}
  \chapter{Titre de chapitre}
  % partout
  \section{Titre de section}
  \subsection{Titre de sous-section}
  \subsubsection{Titre de sous-sous-section}
  \paragraph{Titre de paragraphe}
\end{latex}

\begin{latex}[pdf=False]{sections-star}
  % sans numéro
  \section*{Titre de section non numéroté}
\end{latex}

\begin{frame}{Découpage en sections}
  \pyg{sections}
  \begin{itemize}
    \item numérotation automatique
      \pyg{sections-star}
    \item style adapté par chaque classe
    \item hyper-liens et navigation PDF rapide
      \hfill\minor{(automatique avec lua\LaTeX)}
  \end{itemize}
\end{frame}

%%

\begin{latex}[tpl=standalone,textwidth=10cm]{toc}
  % remplace "Table des matières"
  \def\contentsname{Sommaire}
  \tableofcontents

  \begin{abstract}
    Lorem ipsum dolor sit amet,
    consectetur adipiscing elit.
    Cras condimentum mauris nec
    est tempus viverra.
  \end{abstract}

  \section{Introduction}
  \section{Problématique}
  \subsection{Thèse}
  \subsection{Antithèse}
  \subsection{Synthèse}
  \section{Conclusion}
\end{latex}

\begin{frame}{Sommaires et tables des matières}
  \vspace*{-2mm}
  \relsize{-1}
  \pygpdf[.9\columnwidth]{toc}
\end{frame}

%%

\begin{latex}[tpl=standalone,textwidth=10cm]{refs}
  \section{Introduction}
  Cet article commence à la
  section~\ref{sec:general}\dots

  \section{Présentation générale}
  \label{sec:general}
  Blah, blah\dots

  \subsection{Sujet principal}
  \label{sec:sujet}
  Blah, blah\dots

  \section{Conclusion}
  On a vu des choses à la
  section~\ref{sec:general}
  page~\pageref{sec:general},
  en effet, la
  section~\ref{sec:sujet}\dots
\end{latex}

\begin{frame}{Références croisées}
  \vspace*{-2mm}
  \relsize{-1}
  \pygpdf{refs}
\end{frame}

%%

\begin{latex}[tpl=standalone,textwidth=5cm,latex=pdflatex]{refs-one}
  Prochaine: \ref{sec:ma-section}
  \section{Ma section}
  \label{sec:ma-section}
\end{latex}

\begin{latex}[tpl=standalone,textwidth=5cm,pyg=False]{refs-two}
  Prochaine: \ref{sec:ma-section}
  \section{Ma section}
  \label{sec:ma-section}
\end{latex}

\begin{frame}[fragile]{Résolution des références en avant}
  \begin{multicols}{2}
    \pyg{refs-one}
    \columnbreak
    \begin{itemize}
      \item après une seule compilation
        \pdf[width=5cm]{refs-one}
      \item après deux compilations
        \pdf[width=5cm]{refs-two}
    \end{itemize}
  \end{multicols}
\end{frame}

%%

\begin{latex}[pdf=False]{bib-cite}
  L'article~\cite{MonArticle} en résume
  deux autres~\cite{Article1,Article2}.
\end{latex}

\begin{latex}[pdf=False]{bib-make}
  \bibliographystyle{plain}
  \bibliography{mabiblio}
\end{latex}

\begin{frame}{Bibliographie}
  \begin{itemize}
  \item {\BibTeX} à la rescousse de {\LaTeX}
  \item le fichier {\LaTeX} contient:
    \begin{itemize}
    \item des références bibliographiques:
      \pyg{bib-cite}
    \item un lien vers la bibliographie:
      \pyg{bib-make}
    \end{itemize}
  \item \tt{mabiblio.bib} contient les données bibliographiques
  \item on compile en quatre temps:
    \begin{tabbing}
    \* \tt{latex doc} \quad\=\minor{(mémorise les articles cités)}\\
    \* \tt{bibtex doc} \>\minor{(génère la liste de références)}\\
    \* \tt{latex doc} \>\minor{(insère la liste de références)}\\
    \* \tt{latex doc} \>\minor{(résout les références en avant)}
    \end{tabbing}
  \item si on ne change pas de référence: une seule compilation suffit
  \end{itemize}
\end{frame}

%%
%%

\begin{latex}[tpl=article-crop,textwidth=95mm,files=mabiblio.bib]{bib-plain}
  L'article \cite{MonArticle}
  en synthétise deux
  autres \cite{Article1,Article2}.

  \bibliographystyle{plain}
  \bibliography{mabiblio}
\end{latex}

\begin{latex}[tpl=article-crop,textwidth=95mm,files=mabiblio.bib]{bib-alt}
  L'article \cite{MonArticle}
  en synthétise deux
  autres \cite{Article1,Article2}.

  \bibliographystyle{apalike}
  \bibliography{mabiblio}
\end{latex}

\begin{frame}{Des références et du style}
  \vspace*{-2mm}
  \relsize{-1}
  \begin{multicols}{2}
    \pyg{bib-plain}
    \columnbreak
    \pyg{bib-alt}
  \end{multicols}
  \begin{multicols}{2}
    \pdf{bib-plain}
    \columnbreak
    \pdf{bib-alt}
  \end{multicols}
\end{frame}

%%
%%

\begin{frame}{Fichiers {\BibTeX} (\tt{mabiblio.bib})}
  \let\listingsize=\scriptsize
  \verbinput{mabiblio.bib}
\end{frame}

%%
%%
%%

\section{Figures et images}

%%

\begin{latex}[tpl=article-crop,textwidth=9cm]{figs}
  \begin{figure}
    \centerline{\fbox{Un beau dessin!}}
    \caption{Légende instructive.}
  \end{figure}

  Beaucoup de texte, en quantité
  incroyable, de manière à produire
  au moins une ligne complète.

  \begin{figure}
    \centerline{\fbox{Une photo!}}
    \caption{Légende utile.}
  \end{figure}
\end{latex}

\begin{frame}{Figures}{\tt{=} blocs flottants}
  \begin{multicols}{2}
    \pyg{figs}
    \columnbreak
    \pdf{figs}
    \begin{itemize}
      \item \macro{centerline} \> centre une ligne
      \item \macro{fbox} \> encadre
      \item remarquez l'ordre des trois blocs
    \end{itemize}
  \end{multicols}
\end{frame}

%%

\begin{latex}[pdf=False]{floats}
  \begin{figure}[t] % en haut (si possible)
  ...

  \begin{figure}[h] % ici même (si possible)
  ...

  \begin{figure}[b] % en bas (si possible)
  ...

  \begin{figure}[p] % sur une page complète (si possible)
  ...

  \begin{figure}[t!] % en haut quoi qu'il arrive
  ...

  \begin{figure}[tb] % en haut si possible, sinon en bas
                     % (sinon débrouille toi)
  ...
\end{latex}

\begin{frame}{Placement des éléments flottants}
  \relsize{-1}
  \pyg{floats}
\end{frame}

%%

\begin{latex}[tpl=standalone,textwidth=9cm,files=pix/logo-*.pdf]{pics}
  % \usepackage{graphicx}
  \includegraphics[width=4cm,keepaspectratio=true]{pix/logo-ueve}
  \hspace{1cm} % espace horizontale
  \includegraphics[width=4cm,keepaspectratio=true]{pix/logo-ibisc}
\end{latex}

\begin{frame}{Inclusion d'images externes}
  \pyg{pics}
  \vfill
  \centerline{\pdf[scale=.6]{pics}}
\end{frame}

%%

\begin{latex}[tpl=standalone,textwidth=8cm]{tikz-inline}
  Dessin \tikz\draw[rounded corners](0,0)--(1,.5)--(2,0); dans le texte.
\end{latex}

\begin{latex}[tpl=standalone,textwidth=4cm]{tikz-block}
  \begin{tikzpicture}[xscale=2]
    \node[draw,circle,very thick] (A) at (0,0) {$A$};
    \node[draw,circle,densely dotted] (B) at (1,1) {$B$};
    \draw[->](A)--(B);
  \end{tikzpicture}
\end{latex}

\begin{frame}{Dessins avec \TikZ}
  \pyg{tikz-inline}
  \vfill
  \centerline{\pdf[width=.5\textwidth]{tikz-inline}}
  \vfill
  \pyg{tikz-block}
  \centerline{\pdf[width=.25\textwidth]{tikz-block}}
\end{frame}

%%
%%
%%

\end{document}
